\title{Assignment 2: CS 754, Advanced Image Processing}
\author{}
\date{Due: 19th Feb before 11:55 pm}

\documentclass[11pt]{article}

\usepackage{amsmath}
\usepackage{amssymb}
\usepackage{hyperref}
\usepackage{ulem}
\usepackage[margin=0.5in]{geometry}
\begin{document}
\maketitle

\textbf{Remember the honor code while submitting this (and every other) assignment. All members of the group should work on and \emph{understand} all parts of the assignment. We will adopt a \textbf{zero-tolerance policy} against any violation.}
\\
\\
\textbf{Submission instructions:} You should ideally type out all the answers in Word (with the equation editor) or using Latex. In either case, prepare a pdf file. Create a single zip or rar file containing the report, code and sample outputs and name it as follows: A2-IdNumberOfFirstStudent-IdNumberOfSecondStudent.zip. (If you are doing the assignment alone, the name of the zip file is A2-IdNumber.zip). Upload the file on moodle BEFORE 11:55 pm on 19th Feb. No assignments will be accepted after 10 am on 20th Feb. Note that only one student per group should upload their work on moodle. Please preserve a copy of all your work until the end of the semester. \emph{If you have difficulties, please do not hesitate to seek help from me.} 

\begin{enumerate}
\item Prove the following relationship between the restricted isometry constant of order $s$ of a matrix $\boldsymbol{A}$ (denoted as $\delta_S$) and the mutual coherence $\mu$ of $\boldsymbol{A}$: $\delta_s \leq (s-1)\mu$. Assume all columns of $\boldsymbol{A}$ are unit-normalized. Show all steps very carefully. \textsf{[10 points]}

\item Consider the restricted isometry constants of a matrix $A$, of orders $s_1$ and $s_2$ where $s_1 < s_2$. Which of the following is true and why? (1) \textit{For any matrix $A$, we have} $\delta_{s_1} \leq  \delta_{s_2}$; (2) \textit{For any matrix $A$, we have} $\delta_{s_1} \geq \delta_{s_2}$; (3) The inequality cannot be exactly determined. \textsf{[10 points]}

\item Your task here is to implement the ISTA algorithm for the following three cases:
\begin{enumerate}
\item Consider the `Barbara' image from the homework folder. Add iid Gaussian noise of mean 0 and variance 4 (on a [0,255] scale) to it, using the `randn' function in MATLAB. Thus $\boldsymbol{y} = \boldsymbol{x} + \boldsymbol{\eta}$ where $\boldsymbol{\eta} \sim \mathcal{N}(0,4)$. You should obtain $\boldsymbol{x}$ from $\boldsymbol{y}$ using the fact that patches from $\boldsymbol{x}$ have a sparse or near-sparse representation in the 2D-DCT basis. 
\item Divide the image shared in the homework folder into patches of size $8 \times 8$. Let $\boldsymbol{x_i}$ be the vectorized version of the $i^{th}$ patch. Consider the measurement $\boldsymbol{y_i} = \boldsymbol{\Phi x_i}$ where $\boldsymbol{\Phi}$ is a $32 \times 64$ matrix with entries drawn iid from $\mathcal{N}(0,1)$. Note that $\boldsymbol{x_i}$ has a near-sparse representation in the 2D-DCT basis $\boldsymbol{U}$ which is computed in MATLAB as `kron(dctmtx(8)',dctmtx(8)')'. In other words, $\boldsymbol{x_i} = \boldsymbol{U \theta_i}$ where $\boldsymbol{\theta_i}$ is a near-sparse vector. Your job is to reconstruct each $\boldsymbol{x_i}$ given $\boldsymbol{y_i}$ and $\boldsymbol{\Phi}$ using ISTA. Then you should reconstruct the image by averaging the overlapping patches. You should choose the $\alpha$ parameter in the ISTA algorithm judiciously. Choose $\lambda = 1$ (for a [0,255] image). Display the reconstructed image in your report. State the RMSE given as $\|X(:)-\hat{X}(:)\|_2/\|X(:)\|_2$ where $\hat{X}$ is the reconstructed image and $X$ is the true image. Repeat this with the `goldhill' image (take the top-left portion of size 256 by 256 only). \textsf{[20 points]}
\item Repeat the reconstruction task (for both images) using the Haar wavelet basis via the MATLAB command `dwt2' with the option `db1'. Display the reconstructed image in your report. State the RMSE. Use MATLAB function handles carefully. \textsf{[8 points]}
\item Consider a 100-dimensional sparse signal $\boldsymbol{x}$ containing 10 non-zero elements. Let this signal be convolved with a kernel $\boldsymbol{h} = [1,2,3,4,3,2,1]/16$ followed by addition of Gaussian noise of standard deviation equal to 5\% of the magnitude of $\boldsymbol{x}$ to yield signal $\boldsymbol{y}$, i.e. $\boldsymbol{y} = \boldsymbol{h}*\boldsymbol{x} + \boldsymbol{\eta}$. Your job is to reconstruct $\boldsymbol{x}$ from $\boldsymbol{y}$ given $\boldsymbol{h}$. Be careful of how you create the matrix $\boldsymbol{A}$ in the ISTA algorithm. \textsf{[7 points]}
\end{enumerate}

\item Refer to a copy of the paper `The restricted isometry property and its implications for compressed sensing' in the homework folder. Your task is to open the paper and answer the question posed in each and every green-colored highlight. The task is essentially the complete proof of Theorem 3 done in class. \textsf{[30 points = 2 points for each of the 15 questions]}

\item We have studied two greedy algorithms for compressive recovery in class - MP and OMP. Your task is to do a google search and find out a research paper that proposes a greedy algorithm for CS recovery (that is different from OMP and MP) \emph{and} proves some performance bounds regarding the algorithm. Write down the paper title, author list, venue and year of publication in your report. Also write down the main algorithm from the paper in your report in the form of a simple pseudo-code. State the key theorem from the paper which presents performance bounds for the algorithm, and explain the meaning of the terms involved. If there are multiple theorems, pick the one that states the strongest result. \textsf{[10+5=15 points]}

\end{enumerate}
\end{document}